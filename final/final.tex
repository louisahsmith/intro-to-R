\documentclass[11pt]{scrartcl}
\usepackage{ifxetex,ifluatex}
\usepackage{fixltx2e} % provides \textsubscript

% use upquote if available, for straight quotes in verbatim environments
\IfFileExists{upquote.sty}{\usepackage{upquote}}{}
% use microtype if available
\IfFileExists{microtype.sty}{%
\usepackage{microtype}
\UseMicrotypeSet[protrusion]{basicmath} % disable protrusion for tt fonts
}{}
\usepackage{hyperref}
\hypersetup{unicode=true,
            pdftitle={Final challenge},
            pdfborder={0 0 0},
            breaklinks=true}
\urlstyle{same}  % don't use monospace font for urls

\usepackage{color}
\usepackage{fancyvrb}
\newcommand{\VerbBar}{|}
\newcommand{\VERB}{\Verb[commandchars=\\\{\}]}
\DefineVerbatimEnvironment{Highlighting}{Verbatim}{commandchars=\\\{\}}
% Add ',fontsize=\small' for more characters per line
\usepackage{framed}
\definecolor{shadecolor}{RGB}{248,248,248}
\newenvironment{Shaded}{\begin{snugshade}}{\end{snugshade}}
\newcommand{\AlertTok}[1]{\textcolor[rgb]{0.94,0.16,0.16}{#1}}
\newcommand{\AnnotationTok}[1]{\textcolor[rgb]{0.56,0.35,0.01}{\textbf{\textit{#1}}}}
\newcommand{\AttributeTok}[1]{\textcolor[rgb]{0.77,0.63,0.00}{#1}}
\newcommand{\BaseNTok}[1]{\textcolor[rgb]{0.00,0.00,0.81}{#1}}
\newcommand{\BuiltInTok}[1]{#1}
\newcommand{\CharTok}[1]{\textcolor[rgb]{0.31,0.60,0.02}{#1}}
\newcommand{\CommentTok}[1]{\textcolor[rgb]{0.56,0.35,0.01}{\textit{#1}}}
\newcommand{\CommentVarTok}[1]{\textcolor[rgb]{0.56,0.35,0.01}{\textbf{\textit{#1}}}}
\newcommand{\ConstantTok}[1]{\textcolor[rgb]{0.00,0.00,0.00}{#1}}
\newcommand{\ControlFlowTok}[1]{\textcolor[rgb]{0.13,0.29,0.53}{\textbf{#1}}}
\newcommand{\DataTypeTok}[1]{\textcolor[rgb]{0.13,0.29,0.53}{#1}}
\newcommand{\DecValTok}[1]{\textcolor[rgb]{0.00,0.00,0.81}{#1}}
\newcommand{\DocumentationTok}[1]{\textcolor[rgb]{0.56,0.35,0.01}{\textbf{\textit{#1}}}}
\newcommand{\ErrorTok}[1]{\textcolor[rgb]{0.64,0.00,0.00}{\textbf{#1}}}
\newcommand{\ExtensionTok}[1]{#1}
\newcommand{\FloatTok}[1]{\textcolor[rgb]{0.00,0.00,0.81}{#1}}
\newcommand{\FunctionTok}[1]{\textcolor[rgb]{0.00,0.00,0.00}{#1}}
\newcommand{\ImportTok}[1]{#1}
\newcommand{\InformationTok}[1]{\textcolor[rgb]{0.56,0.35,0.01}{\textbf{\textit{#1}}}}
\newcommand{\KeywordTok}[1]{\textcolor[rgb]{0.13,0.29,0.53}{\textbf{#1}}}
\newcommand{\NormalTok}[1]{#1}
\newcommand{\OperatorTok}[1]{\textcolor[rgb]{0.81,0.36,0.00}{\textbf{#1}}}
\newcommand{\OtherTok}[1]{\textcolor[rgb]{0.56,0.35,0.01}{#1}}
\newcommand{\PreprocessorTok}[1]{\textcolor[rgb]{0.56,0.35,0.01}{\textit{#1}}}
\newcommand{\RegionMarkerTok}[1]{#1}
\newcommand{\SpecialCharTok}[1]{\textcolor[rgb]{0.00,0.00,0.00}{#1}}
\newcommand{\SpecialStringTok}[1]{\textcolor[rgb]{0.31,0.60,0.02}{#1}}
\newcommand{\StringTok}[1]{\textcolor[rgb]{0.31,0.60,0.02}{#1}}
\newcommand{\VariableTok}[1]{\textcolor[rgb]{0.00,0.00,0.00}{#1}}
\newcommand{\VerbatimStringTok}[1]{\textcolor[rgb]{0.31,0.60,0.02}{#1}}
\newcommand{\WarningTok}[1]{\textcolor[rgb]{0.56,0.35,0.01}{\textbf{\textit{#1}}}}


\providecommand{\tightlist}{%
  \setlength{\itemsep}{0pt}\setlength{\parskip}{0pt}}

\setcounter{secnumdepth}{0}
% Redefines (sub)paragraphs to behave more like sections
\ifx\paragraph\undefined\else
\let\oldparagraph\paragraph
\renewcommand{\paragraph}[1]{\oldparagraph{#1}\mbox{}}
\fi
\ifx\subparagraph\undefined\else
\let\oldsubparagraph\subparagraph
\renewcommand{\subparagraph}[1]{\oldsubparagraph{#1}\mbox{}}
\fi


\setlength{\parindent}{0mm}
\setlength{\parskip}{3mm}

\linespread{1}

%%% Use protect on footnotes to avoid problems with footnotes in titles
\let\rmarkdownfootnote\footnote%
\def\footnote{\protect\rmarkdownfootnote}

\makeatletter
\@ifpackageloaded{subfig}{}{\usepackage{subfig}}
\@ifpackageloaded{caption}{}{\usepackage{caption}}
\captionsetup[subfloat]{margin=0.5em}
\AtBeginDocument{%
\renewcommand*\figurename{Figure}
\renewcommand*\tablename{Table}
}
\AtBeginDocument{%
\renewcommand*\listfigurename{List of Figures}
\renewcommand*\listtablename{List of Tables}
}
\@ifpackageloaded{float}{}{\usepackage{float}}
\floatstyle{ruled}
\@ifundefined{c@chapter}{\newfloat{codelisting}{h}{lop}}{\newfloat{codelisting}{h}{lop}[chapter]}
\floatname{codelisting}{Listing}
\newcommand*\listoflistings{\listof{codelisting}{List of Listings}}
\makeatother

% \setlength{\parskip}{1.5em}
\usepackage [autostyle, english = american]{csquotes}
\MakeOuterQuote{"}
\renewcommand{\thesection}{\Roman{section}}
\renewcommand{\thesubsection}{\alph{subsection}}
\usepackage[table]{xcolor}
\usepackage{textcomp}
\usepackage{booktabs}
\usepackage[margin=1in, footskip=.5in]{geometry}
\usepackage{float}
\usepackage{lipsum}
\usepackage{graphicx}
\usepackage{wrapfig}
% \usepackage{underscore}
\usepackage{enumitem}
\usepackage{amsmath}
\usepackage[colorinlistoftodos]{todonotes}
\usepackage{pgf,tikz}
\usepackage{tkz-tab}
% \usepackage[labelformat=simple]{subcaption}
\usepackage[linewidth = .5pt]{mdframed}
\usepackage{multirow}
\usepackage{nicefrac}
\usepackage[nocompress]{cite}
\usepackage{amssymb}
\usepackage{amsbsy}
\usepackage{bm}
\usepackage{array}
\usepackage{anyfontsize}
\usepackage{bold-extra}
\usepackage{bibentry}
\usepackage{longtable}
\usepackage{pdflscape}
\usepackage[labelfont=bf]{caption}
\usepackage{ragged2e}
% \usepackage{color, colortbl}
\usepackage{pgf,tikz}
\usetikzlibrary{positioning, calc, shapes.geometric, shapes.multipart,
  shapes, arrows.meta, arrows,
  decorations.markings, external, trees}
\usepackage{tkz-tab}
\tikzstyle{observed} = [circle, thick, text centered, draw, minimum height=1in, minimum width=1in, text width=9em, text height = 16pt]
\tikzstyle{end_text} = [thick, text centered]

\renewcommand{\rmdefault}{\sfdefault}

\setkomafont{title}{\normalfont\upshape\bfseries}
% \setkomafont{section}{\normalfont\large\upshape\bfseries}
% \setkomafont{subsection}{\normalfont\upshape\bfseries}
\allowdisplaybreaks
\usepackage[all]{nowidow}

% \setlength{\parskip}{0.1mm}
\setlist[enumerate]{leftmargin=*}
\usetikzlibrary{positioning}
\usetikzlibrary{shapes}
\usetikzlibrary{arrows.meta}


\setlist[itemize]{leftmargin=*}

\renewcommand\thesubfigure{(\alph{subfigure})}
\makeatletter
\renewcommand*\env@matrix[1][\arraystretch]{%
  \edef\arraystretch{#1}%
  \hskip -\arraycolsep
  \let\@ifnextchar\new@ifnextchar
  \array{*\c@MaxMatrixCols c}}
\makeatother

\newcommand*\xtb{\mathbf{X}_i^T\bm{\beta}}


\newcommand*{\notindep}{\not\!\perp\!\!\!\perp}

\newcommand{\E}{{\operatorname E}}
\newcommand{\Var}{{\operatorname {Var}}}
\newcommand{\Cov}{{\operatorname {Cov}}}
\newcommand{\Cor}{{\operatorname {Cor}}}

\newcommand*{\independent}{\!\perp\!\!\!\perp}

\newcommand*\pr{\text{Pr}}

\newcommand*{\TitleFont}{%
      % \usefont{\encodingdefault}{\rmdefault}{b}{n}%
      \fontsize{17}{10}%
      \selectfont}
\newcommand*{\AuthFont}{%
      % \usefont{\encodingdefault}{\rmdefault}{}{n}%
      \fontsize{13}{12}%
      \selectfont}


\begin{document}

\begin{flushleft}
  \TitleFont{\textbf{Final challenge}}
    \AuthFont{October 18, 2019}
  
  \end{flushleft}
\vspace{-4.5ex}
\par\noindent\rule{.35\textwidth}{0.6pt}

\hypertarget{prepare-your-project}{%
\section{Prepare your project}\label{prepare-your-project}}

\begin{itemize}
\tightlist
\item
  File -\textgreater{} New Project -\textgreater{} New Directory
  -\textgreater{} New Project
\item
  Name it something like NLSY and put it in an appropriate folder on
  your computer
\item
  Within that folder, make new folders as follows:
\end{itemize}

\begin{verbatim}
NLSY/
 ├── NLSY.Rproj
 ├── data/
 │    ├── raw/
 │    └── processed/
 ├── code/
 └── results/
      ├── tables/
      └── figures/
\end{verbatim}

\hypertarget{move-files-around}{%
\section{Prepare the data}\label{move-files-around}}

\begin{itemize}
\tightlist
\item
  Copy and paste \texttt{nlsy.csv} into \texttt{data/raw}.
\item
  Create a new file and save it as \texttt{clean\_data.R}.
\item
  In that file, read in the NLSY data and load any packages you need.
  Make sure you replace any missing values with \texttt{NA}. Hint: there
  are extra missing values in the \texttt{age\_bir} variable. Also, the
  variable names might be useful:
\end{itemize}

\begin{Shaded}
\begin{Highlighting}[]
\NormalTok{colnames_nlsy <-}\StringTok{ }\KeywordTok{c}\NormalTok{(}
  \StringTok{"glasses"}\NormalTok{, }\StringTok{"eyesight"}\NormalTok{, }\StringTok{"sleep_wkdy"}\NormalTok{, }\StringTok{"sleep_wknd"}\NormalTok{,}
  \StringTok{"id"}\NormalTok{, }\StringTok{"nsibs"}\NormalTok{, }\StringTok{"samp"}\NormalTok{, }\StringTok{"race_eth"}\NormalTok{, }\StringTok{"sex"}\NormalTok{, }\StringTok{"region"}\NormalTok{,}
  \StringTok{"income"}\NormalTok{, }\StringTok{"res_1980"}\NormalTok{, }\StringTok{"res_2002"}\NormalTok{, }\StringTok{"age_bir"}
\NormalTok{)}
\end{Highlighting}
\end{Shaded}

\begin{itemize}
\tightlist
\item
  Add factor labels to \texttt{eyesight}, \texttt{sex},
  \texttt{race\_eth}, \texttt{region}, as in earlier slides. Select
  those variables plus \texttt{income}, \texttt{id}, \texttt{nsibs},
  \texttt{age\_bir}, and the sleep variables. Then restrict to complete
  cases and people with incomes \textless{} \$30,000. Make a variable
  for the log of income (replace with \texttt{NA} if income \textless{}=
  0).
\item
  Also in that file, save your new dataset as a \texttt{.rds} file to
  the \texttt{data/processed} folder.
\end{itemize}

\hypertarget{do-some-analysis}{%
\section{Do some exploratory analysis}\label{do-some-analysis}}

\begin{itemize}
\tightlist
\item
  Create a file called \texttt{create\_figure.R}. In this file, read in
  the cleaned dataset. Load any packages you need. Then make a
  \texttt{ggplot} figure of your choosing to show something about the
  distribution of the data. Save it to the \texttt{results/figures}
  folder as a \texttt{.png} file using the \texttt{ggsave()} function.
\item
  Create a file called \texttt{table\_1.R}. In this file, read in the
  cleaned dataset and use the \texttt{tableone} package to create a
  table 1 with the variables of your choosing. Modify the following code
  to save it as a \texttt{.csv} file. Open it in Excel/Numbers/Google
  Sheets/etc. to make sure it worked.
\end{itemize}

\begin{Shaded}
\begin{Highlighting}[]
\NormalTok{tab1 <-}\StringTok{ }\KeywordTok{CreateTableOne}\NormalTok{(...) }\OperatorTok\StringTok{ }\KeywordTok{print}\NormalTok{() }\OperatorTok\StringTok{ }\KeywordTok{as_tibble}\NormalTok{(}\DataTypeTok{rownames =} \StringTok{"id"}\NormalTok{)}
\KeywordTok{write_csv}\NormalTok{(tab1, ...)}
\end{Highlighting}
\end{Shaded}

\hypertarget{do-some-more-analysis}{%
\section{Do some regression analysis}\label{do-some-more-analysis}}

\begin{itemize}
\tightlist
\item
  In another file called \texttt{lin\_reg.R}, read in the data and run
  the following linear regression:
  \texttt{lm(log\_inc\ \textasciitilde{}\ age\_bir\ +\ sex\ +\ race\_eth\ +\ nsibs,\ data\ =\ nlsy)}.
  Modify the CI function to produce a table of results for a
  \emph{linear} regression. Add an argument \texttt{digits\ =}, with a
  default of 2, to allow you to choose the number of digits you'd like.
  Save it in a separate file called \texttt{functions.R}. Use
  \texttt{source()} to read in the function at the beginning of your
  script.
\item
  Save a table of your results as a \texttt{.csv} file. Make the names
  of the coefficients nice!
\item
  Using the results, use \texttt{ggplot} to make a figure. Use
  \texttt{geom\_point()} for the point estimates and
  \texttt{geom\_errorbar()} for the confidence intervals. It will look
  something like this:
\end{itemize}

\begin{Shaded}
\begin{Highlighting}[]
\KeywordTok{ggplot}\NormalTok{(data) }\OperatorTok{+}\StringTok{ }
\StringTok{  }\KeywordTok{geom_point}\NormalTok{(}\KeywordTok{aes}\NormalTok{(}\DataTypeTok{x =}\NormalTok{ , }\DataTypeTok{y =}\NormalTok{ )) }\OperatorTok{+}\StringTok{ }
\StringTok{  }\KeywordTok{geom_errorbar}\NormalTok{(}\KeywordTok{aes}\NormalTok{(}\DataTypeTok{x =}\NormalTok{ , }\DataTypeTok{ymin =}\NormalTok{ , }\DataTypeTok{ymax =}\NormalTok{ ))}
\end{Highlighting}
\end{Shaded}

\begin{itemize}
\tightlist
\item
  Save that figure as a \texttt{.pdf} using \texttt{ggsave()}. You may
  want to play around with the \texttt{height\ =} and \texttt{width\ =}
  arguments to make it look like you want.
\end{itemize}


\end{document}
